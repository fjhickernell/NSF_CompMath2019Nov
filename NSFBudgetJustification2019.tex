%Fred's and Sou-Cheng's grant 2018
\documentclass[11pt]{NSFamsart}
\usepackage{latexsym,amsfonts,amsmath,epsfig,multirow}
\usepackage{stackrel,tabularx,mathtools,enumitem,booktabs,xspace}
\usepackage[dvipsnames]{xcolor}
\usepackage[numbers]{natbib}
\usepackage{hyperref,accents}
\usepackage[margin=1in]{geometry}

% This package prints the labels in the margin
%\usepackage[notref,notcite]{showkeys}


%\pagestyle{empty}
\thispagestyle{plain}
\pagestyle{plain}

%\headsep-0.6in
%\headsep-0.45in

\textwidth6.4in
\setlength{\oddsidemargin}{0in}
\setlength{\evensidemargin}{0in}
\textheight8.9in
%\textheight9.1in

\begin{document}
%\setlength{\leftmargini}{2.5ex}

\centerline{\textbf{\Large Budget Justification}}

\bigskip

\noindent\textbf{Senior Personnel\footnote{For purposes of NSF PAPPG section II.C.2.g(i)(a), the term “year” at Illinois Institute of Technology refers to IIT’s fiscal year (June 1 -- May 31)}.}
Fred J. Hickernell, the PI, is budgeted at one month of summer salary per year.  He will lead the research along with Mac Hyman, the PI from Tulane University.  This includes co-leading the weekly research seminar and  meeting individually with the 
students and collaborators involved in this project.

Yuhan Ding, the co-PI, is budget at one month of summer salary per year. She will participate in carrying out the research and and assist in supervising the students.

\subsection*{Other Personnel}
The proposed salary for the graduate research assistant is vital, not only for the training of students, 
but also as brainpower to contribute to the construction, analysis, and implementation of our new 
algorithms.  The graduate research assistant will also help to mentor the undergraduate students.

The undergraduate salaries are for two students per summer.  This research experience helps 
prepare students for further study and STEM careers by exposing them to the solving of problems 
whose answers are yet unknown.

\subsection*{Fringe Benefits}
IIT’s federally negotiated fringe benefit rates are: faculty academic salary, 25.2\%; faculty summer salary, 7.7\%; staff salary, 27.4\%; and student stipends, 0.0\%.

\subsection*{Travel}
The PI and co-PI will travel to several conferences each year to disseminate their work and to learn 
what others have discovered.  These include, e.g., annual meetings of the major mathematical and 
statistical societies, specialized meetings in Monte Carlo methods and related fields, and regional 
conferences. 

We often take our students to conferences with us so that they may  experience 
presenting their work to others and network with the academic community.

Some travel funding (about 20\% of what is budgeted) will be used to host visitors who are collaborating on this project.

\subsection*{Other Direct Costs - Tuition}
The current tuition rate is \$1,530 per credit hour. Amount requested is 
\$1,530 $\times$ 9 credit hours per year = \$13,770 per year.

\subsection*{Indirect Costs}
IIT’s current federally negotiated indirect cost rate (agreement date 03/10/2018) is 53\% of modified total direct costs (MTDC). MTDC include all salaries and wages, fringe benefits, materials, supplies, services, travel and up to the first \$25,000 of each subaward. MTDC excludes equipment, capital expenditures, student tuition, rental costs of off-site facilities, as well as the portion of each subaward in excess of \$25,000.

\bigskip

\centerline{
\begin{tabular}{l@{\qquad}r@{\qquad}r@{\qquad}r@{\qquad}r}
    & \multicolumn{1}{c@{\qquad}}{Year 1} & \multicolumn{1}{c@{\qquad}}{Year 2} &\multicolumn{1}{c@{\qquad}}{Year 3} & \multicolumn{1}{c}{Total}  \\
    \toprule
    Direct Costs & \$85\,016 & \$88\,416 & \$91\,954 & \$265\,386 \\
    Indirect Costs & \$37\,760 & \$39\,270 & \$40\,842 & \$117\,872\\
    Total Costs & \$122\,776& \$127\,686 & \$132\,796 & \$383\,258\\
    \emph{Modified Base} & \$71\,246 & \$74\,095 & \$77\,060 & \$222\,401 \\
\end{tabular}}
\bigskip
\noindent An inflationary rate of 4\% is used for all categories for all years of the project.

\end{document} 